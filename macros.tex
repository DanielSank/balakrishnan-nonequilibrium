\newcommand{\angavg}[1]{\left \langle #1 \right \rangle}
\newcommand{\baravg}[1]{\overline{#1}}
\newcommand{\abs}[1]{\left \lvert #1 \right \rvert}
\newcommand{\ket}[1]{\left \lvert #1 \right \rangle}
\newcommand{\bra}[1]{\left \langle #1 \right \rvert}
\newcommand{\bbraket}[3]{\left \langle #1 \left \lvert #2 \right \rvert #3 \right \rangle}

% Figures. Example usage:
% \quickfig{\columnwidth}{my_image}{This is the caption}{fig:my_fig}
\DeclareRobustCommand{\quickfig}[4]{
\begin{figure}
\begin{centering}
\includegraphics[width=#1]{#2}
\par\end{centering}
\caption{#3}
\label{#4}
\end{figure}
}

\DeclareRobustCommand{\quickwidefig}[4]{
\begin{figure*}[h]
\begin{centering}
\includegraphics[width=#1]{#2}
\par\end{centering}
\caption{#3}
\label{#4}
\end{figure*}
}
