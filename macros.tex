\newcommand{\angavg}[1]{\left \langle #1 \right \rangle}
\newcommand{\baravg}[1]{\overline{#1}}
\newcommand{\abs}[1]{\left \lvert #1 \right \rvert}
\newcommand{\ket}[1]{\left \lvert #1 \right \rangle}
\newcommand{\bra}[1]{\left \langle #1 \right \rvert}
\newcommand{\bbraket}[3]{\left \langle #1 \left \lvert #2 \right \rvert #3 \right \rangle}

\newcommand{\tavg}[1]{{\langle\!~#1~\!\rangle_\text{eq}}}
\newcommand{\avg}[1]{{\langle\!~#1~\!\rangle}}  
\newcommand{\pt}[1]{{\frac{\partial}{\partial t}~#1}}
\newcommand{\pv}[1]{{\frac{\partial #1}{\partial v}}}
\newcommand{\p}[2]{{\frac{\partial #1}{\partial #2}}}
\newcommand{\ptwo}[2]{{\frac{\partial^2 #1}{\partial #2^2}}}
\newcommand{\laplace}[1]{{\int_{0}^{\infty}dt~#1 e^{-s t}}}
\newcommand{\ft}[2]{{\int_{-\infty}^{\infty} d #1~ #2 e^{-i k #1}}}
\newcommand{\braket}[2]{{\langle #1|#2 \rangle}}
\newcommand{\cumulant}[1]{\left \langle \left \langle #1 \right \rangle \right \rangle}


% Figures. Example usage:
% \quickfig{\columnwidth}{my_image}{This is the caption}{fig:my_fig}
\DeclareRobustCommand{\quickfig}[4]{
\begin{figure}
\begin{centering}
\includegraphics[width=#1]{#2}
\par\end{centering}
\caption{#3}
\label{#4}
\end{figure}
}

\DeclareRobustCommand{\quickwidefig}[4]{
\begin{figure*}[h]
\begin{centering}
\includegraphics[width=#1]{#2}
\par\end{centering}
\caption{#3}
\label{#4}
\end{figure*}
}
