\levelstay{5.6.3}

\leveldown{Problem}

\levelstay{Solution}

\leveldown{Part a}
We need to formally solve the integral equation
\begin{equation}
  \tilde{p}(\xi, s)
  = \frac{\delta(\xi - \xi_0)}{s + \lambda(\xi)}
  + \frac{\lambda(\xi) p(\xi)}{\angavg{\lambda} (s + \lambda(\xi))}
  \int d\xi' \, \lambda(\xi') \tilde{p}(\xi', s)
\end{equation}
This is an especially simple form of a Fredholm equation.
To solve it, we just multiply both sides of the equation by $\lambda(\xi)$ and integrate over $\xi$:
\begin{align}
  \underbrace{\int d\xi \, \lambda(\xi) \tilde{p}(\xi, s)}_\mathcal{I}
  &= \int d\xi \, \frac{\lambda(\xi) \delta(\xi - \xi_0)}{s + \lambda(\xi)}
  + \int d\xi \frac{\lambda(\xi)^2 p(\xi)}{\angavg{\lambda}(s + \lambda(\xi))}
  \underbrace{\int d\xi' \, \lambda(\xi') \tilde{p}(\xi', s)}_\mathcal{I} \nonumber \\
  \mathcal{I}
  \left(
    1 - \int d\xi' \frac{\lambda(\xi')^2p(\xi')}{\angavg{\lambda}(s + \lambda(\xi'))}
  \right)
  &= \frac{\lambda(\xi_0)}{s + \lambda(\xi_0)} \nonumber \\
  \mathcal{I}
  &= \frac{\lambda(\xi_0)}{s + \lambda(\xi_0)}
  \bigg/
  \left(
    1 - \int d\xi' \frac{\lambda(\xi')^2p(\xi')}{\angavg{\lambda}(s + \lambda(\xi'))}
  \right) \nonumber
  \, .
\end{align}
We have successfully re-expressed the integral $\mathcal{I}$, which involves $\tilde{p}$, in terms of an integral that does not involve $\tilde{p}$, so we formally solve the equation simply by inserting $\mathcal{I}$ into the original equation.
\begin{equation}
  \tilde{p}(\xi, s)
  = \frac{\delta(\xi - \xi_0)}{s + \lambda(\xi)}
  + \frac{\lambda(\xi) \lambda(\xi_0) p(\xi)}{\angavg{\lambda}(s + \lambda(\xi))(s + \lambda(\xi_0))\phi(s)}
\end{equation}
where
\begin{equation}
  \phi(s)
  = 1 - \int d\xi' \, \frac{\lambda(\xi)^2 p(\xi')}{\angavg{\lambda} (s + \lambda(\xi'))}
\end{equation}
which is what we wanted to show.

\levelstay{Part b}
We assume $\lambda(\xi) = \text{const.}$ and see if we can recover the result from the previous problem.
If $\lambda(\xi)$ is constant, then
\begin{align}
  \phi(s)
  &= 1 - \int d\xi' \, \frac{p(\xi') \lambda^2}{\angavg{\lambda}(s + \lambda)} \nonumber \\
  &= 1 - \frac{\lambda}{s + \lambda}
  \, .
\end{align}
Sticking that into our solution for $\tilde{p}(\xi, s)$ we get
\begin{align}
  \tilde{p}(\xi, s)
  &= \frac{\delta(\xi - \xi_0)}{s + \lambda}
  + \frac{\lambda^2 p(\xi)}{(s + \lambda)^2 \lambda \left( 1 - \frac{\lambda}{s + \lambda}\right)} \nonumber \\
  &= \frac{\delta(\xi - \xi_0)}{s + \lambda}
  + \frac{\lambda p(\xi)}{(s + \lambda) s} \nonumber \\
  (\text{partial fractions}) \quad
  &= \frac{\delta(\xi - \xi_0)}{s + \lambda} + \frac{p(\xi)}{s} - \frac{p(\xi)}{s + \lambda} \nonumber \\
  (\text{inverse Laplace}) \quad
  p(\xi, t) &= \delta(\xi - \xi_0) e^{-\lambda t} + p(\xi) \left( 1 - e^{-\lambda t} \right) \nonumber
\end{align}
which is what we wanted to show.

\levelstay{Part c}
Now we find the autocorrelation function.
\begin{align}
  \tilde{C}(s)
  &= \frac{1}{\angavg{\xi^2}}
    \left( \int d\xi_0 \int d\xi \right)
    \xi_0 \, \xi \, \tilde{p}(\xi, s) p(\xi_0)
    \nonumber \\
  &= \frac{1}{\angavg{\xi^2}}
    \left( \int d\xi_0 \int d\xi \right)
      \xi_0 \, \xi \, p(\xi_0)
    \left[
      \frac{\delta(\xi - \xi_0)}{s + \lambda(\xi)} + \frac{p(\xi)\lambda(\xi)\lambda(\xi_0)}{\angavg{\lambda}(s + \lambda(\xi))(s + \lambda(\xi_0))\phi(s)}
    \right] \nonumber \\
  &= \frac{1}{\angavg{\xi^2}}
    \left( \int d\xi_0 \int d\xi \right)
    \frac{\xi_0 \, \xi \, p(\xi_0) \, \delta(\xi - \xi_0)}{s + \lambda(\xi)}
    + \frac{\xi_0 \, \xi \, p(\xi_0) p(\xi) \lambda(\xi) \lambda(\xi_0)}{\angavg{\lambda}(s + \lambda(\xi))(s + \lambda(\xi_0))\phi(s)}
    \nonumber \\
  &= \int d\xi_0 \frac{\xi_0^2 p(\xi_0)}{s + \lambda(\xi_0)}
    + \frac{1}{\angavg{\lambda}\phi(s)}
    \underbrace{\left[
      \int d\xi \frac{\xi p(\xi)\lambda(\xi)}{s + \lambda(\xi)}
    \right]^2}_{0\text{ by symmetry}}
  \nonumber \\
  \text{(inverse Laplace)} \quad C(t)
  &= \frac{1}{\angavg{\xi}^2} \int d\xi \, \xi^2 p(\xi) e^{-\lambda(\xi)t} \nonumber
\end{align}

\levelstay{Part d}

Now we suppose that $\lambda(\xi) = a \abs{\xi}^\beta$ where $\beta > 0$ and $a$ is a constant.
We want to show that in the limit $t \rightarrow \infty$ the correlation function goes as $t^{-3/\beta}$.
This is actually quite easy through a simple change of variables.
\begin{align*}
  C(t)
  &= \frac{1}{\angavg{\xi^2}} \int dx \, x^2 p(x) e^{-\lambda(x) t} \nonumber \\
  &\propto \int dx \, x^2 p(x) e^{-a \abs{x}^\beta t} \nonumber \\
  (\text{Assume }p\text{ is even}) \quad
  &\propto \int_0^\infty dx \, x^2 p(x) e^{-a x^\beta t} \, . \nonumber
\end{align*}
Define $y = t \, x^\beta$ so that
\begin{equation*}
  dx = dy \beta^{-1} y^{(1 - \beta )/ \beta} t^{-1/\beta} \, .
\end{equation*}
Then the correlation function is
\begin{align*}
  C(t)
  &\propto t^{-3/\beta} \int_0^\infty dy \, y^{(3 - \beta) / \beta}
    p\left( (y/t)^{1/\beta} \right) e^{-a y}
\end{align*}
which is what we wanted to show if the dependence of the integral vanishes as $t \rightarrow \infty$.
Indeed, as $t$ becomes large, the argument of the $p$ function becomes significantly different from zero only when $y$ is also large.
However, contributions to the integral from large $y$ are suppressed by the exponential factor.
Therefore, as $t \rightarrow \infty$, we can approximate
\begin{equation*}
  C(t) \propto t^{-3 / \beta} \int_0^\infty dy \, y^{(3 - \beta) / \beta} p(0) e^{-a y} \propto t^{-3 / \beta}
\end{equation*}
as we wanted to show.
