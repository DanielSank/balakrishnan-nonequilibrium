\levelstay{5.6.3}

\leveldown{Problem}

\levelstay{Solution}

For part a) we need to formally solve the integral equation
\begin{equation}
  \tilde{p}(\xi, s)
  = \frac{\delta(\xi - \xi_0)}{s + \lambda(\xi)}
  + \frac{\lambda(\xi) p(\xi)}{\angavg{\lambda} (s + \lambda(\xi))}
  \int d\xi' \, \lambda(\xi') \tilde{p}(\xi', s)
\end{equation}
This is an especially simple form of a Fredholm equation.
To solve it, we just multiply both sides of the equation by $\lambda(\xi)$ and integrate over $\xi$:
\begin{align}
  \underbrace{\int d\xi \, \lambda(\xi) \tilde{p}(\xi, s)}_\mathcal{I}
  &= \int d\xi \, \frac{\lambda(\xi) \delta(\xi - \xi_0)}{s + \lambda(\xi)}
  + \int d\xi \frac{\lambda(\xi)^2 p(\xi)}{\angavg{\lambda}(s + \lambda(\xi))}
  \underbrace{\int d\xi' \, \lambda(\xi') \tilde{p}(\xi', s)}_\mathcal{I} \nonumber \\
  \mathcal{I}
  \left(
    1 - \int d\xi' \frac{\lambda(\xi')^2p(\xi')}{\angavg{\lambda}(s + \lambda(\xi'))}
  \right)
  &= \frac{\lambda(\xi_0)}{s + \lambda(\xi_0)} \nonumber \\
  \mathcal{I}
  &= \frac{\lambda(\xi_0)}{s + \lambda(\xi_0)}
  \bigg/
  \left(
    1 - \int d\xi' \frac{\lambda(\xi')^2p(\xi')}{\angavg{\lambda}(s + \lambda(\xi'))}
  \right) \nonumber
  \, .
\end{align}
We have successfully re-expressed the integral $\mathcal{I}$, which involves $\tilde{p}$, in terms of an integral that does not involve $\tilde{p}$, so we formally solve the equation simply by inserting $\mathcal{I}$ into the original equation.
\begin{equation}
  \tilde{p}(\xi, s)
  = \frac{\delta(\xi - \xi_0)}{s + \lambda(\xi)}
  + \frac{\lambda(\xi) \lambda(\xi_0) p(\xi)}{\angavg{\lambda}(s + \lambda(\xi))(s + \lambda(\xi_0))\phi(s)}
\end{equation}
where
\begin{equation}
  \phi(s)
  = 1 - \int d\xi' \, \frac{\lambda(\xi)^2 p(\xi')}{\angavg{\lambda} (s + \lambda(\xi'))}
\end{equation}
which is what we wanted to show.

For part b), we assume $\lambda(\xi) = \text{const.}$ and see if we can recover the result from the previous problem.
If $\lambda(\xi)$ is constant, then
\begin{align}
  \phi(s)
  &= 1 - \int d\xi' \, \frac{p(\xi') \lambda^2}{\angavg{\lambda}(s + \lambda)} \nonumber \\
  &= 1 - \frac{\lambda}{s + \lambda}
  \, .
\end{align}
Sticking that into our solution for $\tilde{p}(\xi, s)$ we get
\begin{align}
  \tilde{p}(\xi, s)
  &= \frac{\delta(\xi - \xi_0)}{s + \lambda}
  + \frac{\lambda^2 p(\xi)}{(s + \lambda)^2 \lambda \left( 1 - \frac{\lambda}{s + \lambda}\right)} \nonumber \\
  &= \frac{\delta(\xi - \xi_0)}{s + \lambda}
  + \frac{\lambda p(\xi)}{(s + \lambda) s} \nonumber \\
  (\text{partial fractions}) \quad
  &= \frac{\delta(\xi - \xi_0)}{s + \lambda} + \frac{p(\xi)}{s} - \frac{p(\xi)}{s + \lambda} \nonumber \\
  (\text{inverse Laplace}) \quad
  p(\xi, t) &= \delta(\xi - \xi_0) e^{-\lambda t} + p(\xi) \left( 1 - e^{-\lambda t} \right) \nonumber
\end{align}
which is what we wanted to show.
