\section{Ch. 5 (AO)}
\subsection{Problem 5.6.1}

\textbf{(a):} Define the transition rate from state $\xi_1 \rightarrow \xi_2$ as $\lambda_1=1/\tau_1$, and the transition rate from state $\xi_2 \rightarrow \xi_1$ as $\lambda_2=1/\tau_2$. Then
\begin{eqnarray}
\pt p(\xi_1, t|\xi_0) &=& \lambda_2 p(\xi_2, t|\xi_0) - \lambda_1 p(\xi_1, t|\xi_0) \nonumber \\
\pt p(\xi_2, t|\xi_0) &=& \lambda_1 p(\xi_1, t|\xi_0) - \lambda_2 p(\xi_2, t|\xi_0), \nonumber
\end{eqnarray}
where $\xi_0$ is the initial state. Noting that $p(\xi_1, t| \xi_0) + p(\xi_2, t| \xi_0) = 1$ and taking the limit as $t\rightarrow \infty$ yields \footnote{We assume that $\lim_{t\rightarrow \infty} p(\xi_i, t|\xi_0) = p(\xi_i)$.}
\begin{eqnarray}
p(\xi_1) &=& \frac{\lambda_2}{(\lambda_2 + \lambda_1)} = \frac{\tau_1}{\tau_1 + \tau_2} \nonumber \\
p(\xi_2) &=& \frac{\lambda_1}{(\lambda_2 + \lambda_1)} = \frac{\tau_2}{\tau_1 + \tau_2}. \nonumber
\end{eqnarray}
The expectation value of $\xi$ is then
\begin{eqnarray}
\avg{\xi} &=& \sum_{i=1}^{2} \xi_i p(\xi_i) \nonumber \\
&=& \xi_1 \times \frac{\tau_1}{\tau_1 + \tau_2} + \xi_2 \times \frac{\tau_2}{\tau_1 + \tau_2} \nonumber \\
&=& \boxed{\frac{\tau_1 \xi_1 + \tau_2 \xi_2}{\tau_1 + \tau_2}}~. \nonumber
\end{eqnarray}
The standard deviation of $\xi$ is
\begin{eqnarray}
\avg{(\xi-\avg{\xi})^2} &=& \sum_{i=1}^{2} (\xi_i - \avg{\xi})^2 p(\xi_i) \nonumber \\ 
&=& \bigg(\xi_1 - \frac{\tau_1 \xi_1 + \tau_2 \xi_2}{\tau_1 + \tau_2}\bigg)^2 \times \frac{\tau_1}{\tau_1 + \tau_2} + \bigg(\xi_2 - \frac{\tau_1 \xi_1 + \tau_2 \xi_2}{\tau_1 + \tau_2}\bigg)^2 \times \frac{\tau_2}{\tau_1 + \tau_2} \nonumber \\
&=& \bigg(\frac{\tau_2 \xi_1 - \tau_2 \xi_2}{\tau_1 + \tau_2}\bigg)^2 \times \frac{\tau_1}{\tau_1 + \tau_2} + \bigg(\frac{\tau_1 \xi_2 - \tau_1 \xi_1}{\tau_1 + \tau_2}\bigg)^2 \times \frac{\tau_2}{\tau_1 + \tau_2} \nonumber \\
&=& \frac{(\xi_1-\xi_2)^2}{(\tau_1 + \tau_2)^3} \times ( \tau_1^2 \tau_2 + \tau_1 \tau_2^2)~. \nonumber \\
&=& \boxed{\frac{\tau_1 \tau_2 (\xi_1-\xi_2)^2}{(\tau_1 + \tau_2)^2}} ~. \nonumber \\
\end{eqnarray}

\textbf{(b):} Using the equations
\begin{equation}
\textbf{W} = \left[\begin{array}{cc} -\lambda_1 & \lambda_2 \\ \lambda_1 & -\lambda_2 \end{array}\right]
\end{equation}
and
\begin{equation}
\textbf{W}^2 = -(\lambda_1 + \lambda_2) \textbf{W}
\end{equation}
we find that
\begin{eqnarray}
e^{\textbf{W}t} &=& \textbf{1} + (\textbf{W}t) + \frac{(\textbf{W}t)^2}{2!} + \frac{(\textbf{W}t)^3}{3!} + \frac{(\textbf{W}t)^4}{4!} + \cdots \nonumber \\
&=& \textbf{1} + \textbf{W} \times \bigg( t - \frac{(\lambda_1 + \lambda_2)t^2}{2!} + \frac{(\lambda_1 + \lambda_2)^2 t^3}{3!} - \frac{(\lambda_1 + \lambda_2)^3 t^4}{4!} + \cdots \bigg) \nonumber \\
&=& \textbf{1} + \frac{\textbf{W}}{(\lambda_1 + \lambda_2)} \times \bigg(1-\exp(-t(\lambda_1 + \lambda_2))\bigg) \nonumber \\
&=& \frac{1}{2\lambda} \left[\begin{array}{cc} \lambda_2 + \lambda_1 e^{-2 \lambda t} & \lambda_2 (1-e^{-2 \lambda t}) \\ \lambda_1 (1-e^{-2 \lambda t}) & \lambda_1 + \lambda_2 e^{-2 \lambda t} \end{array}\right].
\end{eqnarray}
From the relation $\text{P}(t) = e^{\textbf{W}t} \text{P}(0)$, the conditional probabilities are
\begin{eqnarray}
P(\xi_1, t|\xi_1) &=& \left[\begin{array}{cc} 1 & 0  \end{array}\right] e^{\textbf{W}t} \left[\begin{array}{c} 1 \\ 0  \end{array}\right] = (\lambda_2 + \lambda_1 e^{-2 \lambda t}) / (\lambda_1 + \lambda_2) \\
P(\xi_2, t|\xi_1) &=& \left[\begin{array}{cc} 0 & 1  \end{array}\right] e^{\textbf{W}t} \left[\begin{array}{c} 1 \\ 0  \end{array}\right] = \lambda_1 (1-e^{-2 \lambda t}) / (\lambda_1 + \lambda_2) \\
P(\xi_1, t|\xi_2) &=& \left[\begin{array}{cc} 1 & 0  \end{array}\right] e^{\textbf{W}t} \left[\begin{array}{c} 0 \\ 1  \end{array}\right] = \lambda_2 (1-e^{-2 \lambda t}) / (\lambda_1 + \lambda_2) \\
P(\xi_2, t|\xi_2) &=& \left[\begin{array}{cc} 0 & 1  \end{array}\right] e^{\textbf{W}t} \left[\begin{array}{c} 0 \\ 1  \end{array}\right] = (\lambda_1 + \lambda_2 e^{-2 \lambda t}) / (\lambda_1 + \lambda_2).
\end{eqnarray}
Taking the limit as $t \rightarrow \infty$ once again shows that  $p(\xi_1) = \lambda_2/(\lambda_1 + \lambda_2)$ and $p(\xi_2) = \lambda_1/(\lambda_1 + \lambda_2)$.

\textbf{(c):}
\begin{eqnarray}
\avg{\delta \xi(0) \delta \xi(t)} + \avg{\xi}^2 &=& \sum_{i=1}^{2} \xi_1 p(\xi_1) \sum_{j=1}^{2}  \xi_j p(\xi_j, t| \xi_i) \nonumber \\
&=& \frac{1}{(\lambda_1 + \lambda_2)}\sum_{j=1}^{2} \bigg\{  \xi_1 \lambda_2 \xi_j p(\xi_j, t| \xi_1) + \xi_2 \lambda_1 \xi_j p(\xi_j, t| \xi_2) \bigg\} \nonumber \\
&=& \frac{1}{(\lambda_1 + \lambda_2)^2} \bigg\{  \xi_1^2 \lambda_2 (\lambda_2 + \lambda_1 e^{-2 \lambda t}) + \xi_2 \lambda_1 \xi_1 \lambda_2 (1-e^{-2 \lambda t}) \nonumber \\
&+& \xi_1 \lambda_2 \xi_2 \lambda_1 (1-e^{-2 \lambda t}) + \xi_2^2 \lambda_1 (\lambda_1 + \lambda_2 e^{-2 \lambda t}) \bigg\} \nonumber \\
&=& \frac{1}{(\lambda_1 + \lambda_2)^2} \bigg\{ (\lambda_2 \xi_1 + \lambda_1 \xi_2)^2  + \lambda_1 \lambda_2 (\xi_1-\xi_2)^2 e^{- 2 \lambda t} \bigg\} \nonumber  \\
&=& \frac{\lambda_1 \lambda_2 (\xi_1-\xi_2)^2}{(\lambda_1 + \lambda_2)^2} \times e^{- 2 \lambda t} + \avg{\xi}^2
\end{eqnarray}
Therefore
\begin{equation}
\boxed{\avg{\delta \xi(0) \delta \xi(t)} = \frac{\lambda_1 \lambda_2 (\xi_1-\xi_2)^2}{(\lambda_1 + \lambda_2)^2} \times e^{- 2 \lambda t}}~.
\end{equation}

\subsection{Problem 5.6.2}


\textbf{(a):} 
Consider a transition rate of the form 
\begin{equation}
w(\xi|\xi') = \lambda u(\xi) \label{eq:kubo_anderson_rate}
\end{equation} 
in the master equation
\begin{equation}
\pt p(\xi, t | \xi_0) = \int d\xi' \bigg\{ p(\xi', t | \xi_0) w(\xi|\xi') -  p(\xi, t, |\xi_0) w(\xi'|\xi) \bigg\}. \nonumber
\end{equation}
This implies that
\begin{equation}
\pt p(\xi, t | \xi_0) = \lambda \times \bigg\{ \int d \xi' [~p(\xi', t | \xi_0) u(\xi) - p(\xi, t, |\xi_0) u(\xi') ]\bigg\}. \label{eq:kubo_anderson_master}
\end{equation}
Taking the limit as $t\rightarrow\infty$ implies that\footnote{We assume that $\lim_{t\rightarrow\infty} p(\xi, t | \xi_0) = p(\xi)$ and $\lim_{t\rightarrow\infty} \pt p(\xi, t | \xi_0) = 0$.}
\begin{equation}
u(\xi) p(\xi') = p(\xi) u(\xi')
\end{equation}  
in other words 
\begin{equation}
u(\xi) = p(\xi).
\end{equation}
Plugging this result back into the master equation yields
\begin{eqnarray}
\pt p(\xi, t | \xi_0) &=& \lambda \times \bigg\{ p(\xi) \int d \xi' ~p(\xi', t | \xi_0) - p(\xi, t, |\xi_0) \int d \xi' p(\xi')  \bigg\} \\
&=&  \lambda p(\xi) - \lambda p(\xi, t, |\xi_0).
\end{eqnarray}
This equation can be solved using the method of integrating factors which leads to a solution of the form
\begin{eqnarray}
p(\xi, t | \xi_0) &=& g(\xi) e^{-\lambda t} \lambda p(\xi) \int_{0}^{t} e^{\lambda(t'-t)} dt'  \nonumber \\
&=& g(\xi) e^{-\lambda t} + p(\xi) e^{\lambda(t'-t)} \bigg|^{t}_{0} \nonumber \\
&=& g(\xi) e^{-\lambda t} + p(\xi) (1-e^{-\lambda t}),
\end{eqnarray}
where $g(\xi)$ is some unknown function of $\xi$ determined by the initial condition at $t=0$. Using the fact that $p(\xi,t=0|\xi_0) = \delta(\xi-\xi_0)$, one finds that $g(\xi) = \delta(\xi-\xi_0)$, therefore
\begin{equation}
\boxed{p(\xi, t | \xi_0)  = \delta(\xi-\xi_0) e^{-\lambda t} + p(\xi) (1-e^{-\lambda t})}~.
\end{equation}

\textbf{(b):} 
From the correlation function (Eq. (5.31) in the textbook) one has that
\begin{eqnarray}
\avg{\delta \xi(0) \delta \xi(t)} &=& \int d \xi_1\int d\xi_2~\xi_1~\xi_2 ~p(\xi_2, t|\xi_1)~p(\xi_1) - \avg{\xi}^2 \nonumber \\
&=& e^{-\lambda t} \int d \xi_1\int d\xi_2~\xi_1~\xi_2 ~p(\xi_1)~\delta(\xi_2-\xi_1)  \nonumber \\
&+&(1-e^{-\lambda t}) \times \int d \xi_1~\xi_1~p(\xi_1) \int d\xi_2~\xi_2 ~p(\xi_2) - \avg{\xi}^2 \nonumber \\
&=& e^{-\lambda t} \times \int d \xi~\xi^2 ~p(\xi) + (1-e^{-\lambda t}) \avg{\xi}^2 - \avg{\xi}^2 \nonumber \\
&=& \boxed{e^{-\lambda t} \underbrace{(\avg{\xi^2} - \avg{\xi}^2)}_{\equiv \avg{(\delta \xi)^2}}}~. \nonumber \\
\end{eqnarray}


\subsection{Problem 5.6.3}


Consider a transition rate of the form 
\begin{equation}
w(\xi|\xi') = u(\xi) \lambda(\xi') \label{eq:kangaroo_rate}
\end{equation} 
in the master equation
\begin{equation}
\pt p(\xi, t | \xi_0) = \int d\xi' \bigg\{ p(\xi', t | \xi_0) w(\xi|\xi') -  p(\xi, t, |\xi_0) w(\xi'|\xi) \bigg\}. \nonumber
\end{equation}
By Eq (\ref{eq:kangaroo_rate}), we have that
\begin{equation}
\pt p(\xi, t | \xi_0) = u(\xi) \int d \xi'~p(\xi', t | \xi_0) ~\lambda(\xi') - p(\xi, t, |\xi_0)~\lambda(\xi) \int d \xi' u(\xi'). \label{eq:kangaroo_master}
\end{equation}
Taking the limit of this equation as $t\rightarrow \infty$ yields\footnote{This assumes that $\lim_{t\rightarrow \infty} p(\xi, t| \xi_0) = p(\xi)$, in other words, the conditional probability $p(\xi, t | \xi_0)$ tends to the stationary PDF $p(\xi)$ at very long times.}
\begin{eqnarray}
\lim_{t\rightarrow \infty} \pt p(\xi, t| \xi_0) &=& 0 \nonumber \\
&=& u(\xi) \underbrace{\int d \xi'~p(\xi') \lambda(\xi')}_{\equiv~\avg{\lambda}} -  p(\xi) \lambda(\xi) \underbrace{\int d \xi' u(\xi')}_{\equiv ~c} \nonumber \\
&=&  u(\xi)\avg{\lambda} -  p(\xi) \lambda(\xi) c. \nonumber
\end{eqnarray}
This implies that
\begin{equation}
u(\xi) = \frac{p(\xi) \lambda(\xi)c}{\avg{\lambda}}. \nonumber \label{eq:kangaroo_u_of_xi}
\end{equation}
Using Eq. (\ref{eq:kangaroo_u_of_xi}), we can rewrite Eq. (\ref{eq:kangaroo_master}) as
\begin{equation}
\pt p(\xi, t | \xi_0) = \bigg( \frac{p(\xi)\lambda(\xi)c}{\avg{\lambda}} \int d \xi'~p(\xi', t | \xi_0)~\lambda(\xi') \bigg) - p(\xi, t | \xi_0) \lambda(\xi) c. \label{eq:kangaroo_simplified_master}
\end{equation}
In what follows, we take $c=1$.

The Laplace transform\footnote{The Laplace transform of $f(t)$ is given by $\tilde{f}(s) \equiv \laplace{f(t)}$.} of Eq. (\ref{eq:kangaroo_simplified_master}) is given by\footnote{We assume that $p(\xi,t=0|\xi_0) = \delta(\xi-\xi_0)$.}
\begin{eqnarray}
\laplace{\bigg(\pt p(\xi, t|\xi_0) \bigg)} &=& p(\xi, t | \xi_0) e^{-s t} \bigg|^{\infty}_{0^-} + s\laplace{p(\xi, t|\xi_0)} \nonumber \\
&=& - \delta(\xi-\xi_0) + s \tilde{p}(\xi, s) \nonumber \\
&=&  \frac{p(\xi)\lambda(\xi)}{\avg{\lambda}} \int d \xi'~\tilde{p}(\xi', s)~\lambda(\xi') - \tilde{p}(\xi, s) \lambda(\xi)  \nonumber.
\end{eqnarray}
Rearranging this equation, one has that
\begin{equation}
- \delta(\xi-\xi_0) + (\lambda(\xi)+s) \tilde{p}(\xi, s) =  \frac{p(\xi)\lambda(\xi)}{\avg{\lambda}}  \int d \xi'~\tilde{p}(\xi', s)~\lambda(\xi'), \nonumber
\end{equation}
in other words
\begin{equation}
\tilde{p}(\xi, s) = \frac{\delta(\xi-\xi_0)}{(\lambda(\xi)+s)} + \frac{1}{\avg{\lambda}} \times \frac{p(\xi)\lambda(\xi)}{(\lambda(\xi)+s))}  \int d \xi'~\lambda(\xi')~\tilde{p}(\xi', s). \label{eq:kangaroo_integral_equation}
\end{equation}

\textbf{(a):} To solve Eq. (\ref{eq:kangaroo_integral_equation}), we note that it is of the form
\begin{equation}
f(\xi) - \lambda \int_{a}^{b} d \xi'~K(\xi, \xi')~f(\xi') = g(\xi), \label{eq:fredholm}
\end{equation}
which is an inhomogenous Fredholm integral equation of the equation of the second kind.
As noted by Balakrishnan, the kernel of this integral equation is separable and of rank one, meaning that
\begin{eqnarray}
K(\xi, \xi') &=& \sum_{j=1}^{\infty} \phi_j(\xi) \psi_j^*(\xi') \nonumber \\
&=& \bigg(\frac{p(\xi)\lambda(\xi)}{(\lambda(\xi)+s))} \bigg) \times \bigg(  \lambda(\xi') \bigg), \nonumber
\end{eqnarray}
where the sum terminates after $j=1$. To be explicit, we have made the following identifications:
\begin{eqnarray}
f(\xi) &=& \tilde{p}(\xi, s), \\
\lambda &=& 1/\avg{\lambda}, \\
g(\xi) &=& \frac{\delta(\xi-\xi_0)}{(\lambda(\xi)+s)} \\
\phi_1(\xi) &=& \frac{\lambda(\xi) p(\xi)c}{(\lambda(\xi)+s)} , \\
\psi_1^*(\xi') &=& \lambda(\xi').
\end{eqnarray}

Converting Eq. (\ref{eq:fredholm}) into bra-ket notation, we have that
\begin{equation}
\ket{f} - \lambda \textbf{K} \ket{f} = (\textbf{I} - \lambda \textbf{K}) \ket{f} = \ket{g}, \label{eq:braket_conversion}
\end{equation}
where 
\begin{equation}
\textbf{I} = \int_{a}^{b} d \xi'~\ket{\xi'} \bra{\xi'}, \nonumber
\end{equation}
\begin{equation}
\textbf{K} = \sum_{j=1}^{r} \ket{\phi_j} \bra{\psi_j}, \label{eq:kernel_operator}
\end{equation}
$K(\xi, \xi') = \bra{\xi} \textbf{K} \ket{\xi'}$, $f(\xi) = \braket{\xi}{f}$, and $g(\xi) = \braket{\xi}{g}$. Substituting Eq. (\ref{eq:kernel_operator}) into Eq. (\ref{eq:braket_conversion}) yields
\begin{equation}
\ket{f} - \lambda \sum_{j=1}^{r} C_j \ket{\phi_j} = \ket{g}, \nonumber
\end{equation}
where $C_j = \braket{\psi_j}{f}$. To determine $C_1$, we take the inner product of this equation with $\bra{\psi_1}$, which gives us
\begin{equation}
C_1 - \lambda \braket{\psi_1}{\phi_1} C_1 = \braket{\psi_1}{g}. \nonumber
\end{equation}
Converting back into the $\xi$-basis, one has that
\begin{equation}
C_1\times \bigg( 1 - \lambda \int d \xi' \psi^*_1(\xi')\phi_1(\xi') \bigg) = \int d \xi' \psi^*_1(\xi') g(\xi'), \nonumber
\end{equation}
or equivalently that
\begin{eqnarray}
C_1\times \bigg( 1 - \frac{1}{\avg{\lambda}} \int d \xi' \frac{\lambda^2(\xi') p(\xi')}{(\lambda(\xi')+s))} \bigg) &=& \int d \xi' \lambda(\xi') \frac{\delta(\xi'-\xi_0)}{(\lambda(\xi')+s)} \nonumber \\
&=& \frac{\lambda(\xi_0) }{(\lambda(\xi_0)+s)}. \nonumber
\end{eqnarray}
Rearranging for $C_1$ gives
\begin{eqnarray}
C_1 &=& \bigg(\frac{\lambda(\xi_0) }{(\lambda(\xi_0)+s)}\bigg) \times \bigg( \frac{1}{ 1 - \frac{1}{\avg{\lambda}} \int d \xi' \frac{\lambda^2(\xi') p(\xi')}{(\lambda(\xi')+s))}} \bigg) \\
&=& \bigg(\frac{\lambda(\xi_0) }{(\lambda(\xi_0)+s)}\bigg) \times \frac{1}{\phi(s)}~.
\end{eqnarray}
The solution for $\tilde{p}(\xi, s)$ is
%\begin{equation}
%\underbrace{\tilde{p}(\xi, s)}_{f(\xi)} = \underbrace{\frac{\delta(\xi-\xi_0)}{s + \lambda(\xi)}}_{g(\xi)} + \underbrace{\frac{1}{\avg{\lambda}}}_{\lambda} \underbrace{ \Bigg[ \bigg( \frac{\lambda(\xi_0) }{s+\lambda(\xi_0)} \bigg)\times \frac{1}{\phi(s)} \Bigg]}_{C_1} \underbrace{\bigg(\frac{\lambda(\xi) p(\xi)}{s+\lambda(\xi)} \bigg)}_{\phi_1(\xi)} \label{eq:kangaroo_final_part_a}
%\end{equation}
\begin{equation}
\boxed{\tilde{p}(\xi, s) = \frac{\delta(\xi-\xi_0)}{(\lambda(\xi)+s)}+ \frac{1}{\avg{\lambda}} \Bigg[ \bigg( \frac{\lambda(\xi_0) }{(\lambda(\xi_0)+s)} \bigg)\times \frac{1}{\phi(s)} \Bigg] \times \frac{\lambda(\xi) p(\xi)}{(\lambda(\xi)+s)} }. \label{eq:kangaroo_final_part_a}
\end{equation}

\textbf{(b):} Letting $\lambda(\xi) = \lambda$ in Eq. (\ref{eq:kangaroo_final_part_a}) yields\footnote{For a Kubo-Anderson process, one can prove that $u(\xi)=p(\xi)$, which implies that $c=\int u(\xi) d\xi =1$.} 
\begin{eqnarray}
\tilde{p}(\xi, s) &=& \frac{\delta(\xi-\xi_0)}{\lambda+s} + \frac{1}{\lambda} \times \bigg(\frac{\lambda}{\lambda+s}\bigg)^2 \times \bigg(\frac{p(\xi)}{1-\frac{\lambda}{(\lambda+s)}} \bigg) \\
&=& \frac{\delta(\xi-\xi_0)}{\lambda+s} + \frac{\lambda}{(\lambda+s)} \times \frac{p(\xi)}{s} \\
&=& \frac{\delta(\xi-\xi_0)}{\lambda+s} + p(\xi) \times \bigg(\frac{1}{s} - \frac{1}{(\lambda+s)} \bigg).
\end{eqnarray}
Recall the following inverse Laplace transforms:
\begin{eqnarray}
\mathcal{L}^{-1}\bigg( \frac{1}{\lambda+s} \bigg) &=& e^{-\lambda t} \nonumber \\
\mathcal{L}^{-1}\bigg( \frac{1}{s} \bigg) &=& 1 \nonumber.
\end{eqnarray}
From these it follows that
\begin{equation}
\boxed{\mathcal{L}^{-1}( \tilde{p}(\xi, s) ) = p(\xi, t|\xi_0) = \delta(\xi-\xi_0) e^{-\lambda t} + p(\xi) \times(1-e^{-\lambda t})}~.
\end{equation}

\textbf{(c):} Using the definition of the autocorrelation function
\begin{equation}
C(t) = \frac{\avg{\xi(0) \xi(t)}}{\avg{\xi^2}} = \frac{\int d \xi_0 \int d \xi~\xi_0~\xi~p(\xi, t|\xi_0)~p(\xi_0)}{ \int d\xi~\xi^2~p(\xi)},
\end{equation}
and taking the Laplace transform yields
\begin{equation}
\tilde{C}(s) = \frac{1}{\avg{\xi^2}} \int d \xi_0 \int d \xi ~\xi_0~\xi~ \tilde{p}(\xi, s)~p(\xi_0). \label{eq:kangaroo_laplace_corr}
\end{equation}
Now assume $\xi \in (-\infty, \infty)$, and that $p(\xi)$ and $\lambda(\xi)$ are even functions. Substituting Eq. (\ref{eq:kangaroo_final_part_a}) into Eq. (\ref{eq:kangaroo_laplace_corr}) gives us
\begin{eqnarray}
\tilde{C}(s) &=& \frac{1}{\avg{\xi^2}} \Bigg\{\int d \xi_0 \int d \xi ~\xi_0~\xi~ \frac{\delta(\xi-\xi_0)}{(\lambda(\xi)+s)} ~p(\xi_0) \\
&+&\frac{1}{\phi(s) \avg{\lambda}}  \underbrace{\int d \xi_0~\xi_0 \bigg( \frac{\lambda(\xi_0) }{(1\lambda(\xi_0)+s)} \bigg) p(\xi_0)}_{=0~(\text{odd integrand})} \underbrace{\int d \xi ~\xi~ \bigg(\frac{\lambda(\xi)}{(1\lambda(\xi)+s)} \bigg) p(\xi)}_{=0~(\text{odd integrand})} \Bigg\} \\
&=& \frac{1}{\avg{\xi^2}} \int d \xi ~\xi^2~ \frac{1}{(1\lambda(\xi)+s)} ~p(\xi).
\end{eqnarray}
Upon taking the inverse Laplace transform, one has that
\begin{equation}
\boxed{C(t) = \frac{1}{\avg{\xi^2}} \int d \xi ~\xi^2~ e^{-\lambda(\xi) t } ~p(\xi)}~. \label{eq:kangaroo_final_part_c}
\end{equation}

\textbf{(d):} Assume $\lambda(\xi) = \alpha |\xi|^\beta$, where $\alpha, \beta>0$. Then
\begin{equation}
C(t) = \frac{1}{\avg{\xi^2}} \int_{-\infty}^{\infty} d \xi ~\xi^2~ e^{-\alpha |\xi|^\beta t}  ~p(\xi) = \frac{2}{\avg{\xi^2}} \int_{0}^{\infty} d \xi ~\xi^2~ e^{-\alpha \xi^\beta t}  ~p(\xi).
\end{equation}
Making the change of variables $z= \alpha \xi^\beta t$, one has that $dz = \alpha \beta \xi^{\beta-1} t d \xi$. Solving for $\xi$ and $d \xi$ in terms of $z$ and $dz$ yields:
\begin{equation}
\xi = \bigg( \frac{z}{\alpha t}\bigg)^\frac{1}{\beta}
\end{equation}
and
\begin{equation}
d\xi = dz \times \frac{1}{\alpha \beta t}\bigg( \frac{z}{\alpha t}\bigg)^\frac{1-\beta}{\beta}.
\end{equation}
Substituting these expressions into the integral above yields

\begin{eqnarray}
C(t) &=& \frac{2}{\avg{\xi^2}} \int_{0}^{\infty} dz \frac{1}{\alpha \beta t}\bigg( \frac{z}{\alpha t}\bigg)^\frac{1-\beta}{\beta} ~\bigg( \frac{z}{\alpha t}\bigg)^\frac{2}{\beta}~ e^{-z}  ~p((z/\alpha t)^\frac{1}{\beta}) \\
&=& \frac{2}{\avg{\xi^2}} \bigg(\frac{1}{\beta} \bigg) \bigg(\frac{1}{\alpha t}  \bigg)^{\frac{3}{\beta}}  \int_{0}^{\infty} z^\frac{3-\beta}{\beta} e^{-z}  ~p((z/\alpha t)^\frac{1}{\beta}) \\
&\propto& t^{-3/\beta}.
\end{eqnarray}
Since $(3-\beta)/\beta > -1$, the integral of $z^{(3-\beta)/ \beta} e^{-z}$ poses no difficulties as $z\rightarrow0$. Similarly, as $z\rightarrow \infty$ this term rapidly decays.

\textbf{(e):} Let us define
\begin{equation}
f(\xi, t) = \xi^2 \exp(-t\lambda(\xi)) 
\end{equation}
such that Eq. (\ref{eq:kangaroo_final_part_c}) can be written as
\begin{equation}
C(t) = \frac{1}{\avg{\xi^2}} \int d \xi~f(\xi, t) ~p(\xi).  \label{eq:kangaroo_correlation_fxn_in_f}
\end{equation}
The maximum of $f(\xi, t)$ is given by
\begin{equation}
\frac{\partial}{\partial \xi} f(\xi, t) = 2 \xi \exp(-t \lambda(\xi)) -  t \lambda'(\xi) \exp(-t \lambda(\xi)) \xi^2 = 0, \nonumber
\end{equation}
which implies
\begin{equation}
2 = \xi  t \lambda'(\xi). \nonumber
\end{equation}
Let us denote the solution of this equation by $\bar{\xi}_t$, i.e., $\bar{\xi}_t t \lambda'(\bar{\xi}_t) = 2$. 

The curvature of $f(\xi, t)$ is given by
\begin{equation}
\frac{\partial^2}{\partial \xi^2} f(\xi, t) = \exp(-t \lambda(\xi)) \times \bigg( 2 - 4 \xi  t \lambda'(\xi)  - t  \lambda''(\xi) \xi^2  + (\xi  t \lambda')^2 \bigg). \nonumber
\end{equation}
Evaluating this function at $\bar{\xi}_t$ yields
\begin{equation}
\frac{\partial^2}{\partial \xi^2} f(\xi, t) \bigg|_{\bar{\xi}_t} =  \exp(- t \lambda(\bar{\xi}_t)) \times \bigg(-2 -  t \lambda''(\bar{\xi}_t) \bar{\xi}_t^2\bigg). \nonumber
\end{equation}
Approximating $f(\xi, t)$ as a Taylor expansion about $\bar{\xi}_t$ shows that
\begin{eqnarray}
f(\xi, t)  &\approx &  \bar{\xi}_t^2 \exp(- t \lambda(\bar{\xi}_t)) \times \bigg\{1 -\bigg[2/\bar{\xi}_t^2 +  t \lambda''(\bar{\xi}_t)\bigg] \times (\xi - \bar{\xi}_t)^2  \bigg\} + \mathcal{O}((\xi - \bar{\xi}_t)^3) \nonumber \\
&\approx & \bar{\xi}_t^2 \exp(- t \lambda(\bar{\xi}_t)) \times \exp\bigg(-\bigg[2/\bar{\xi}_t^2 +  t \lambda''(\bar{\xi}_t)\bigg] \times (\xi - \bar{\xi}_t)^2\bigg). \label{eq:kangaroo_gaussian_approx}
\end{eqnarray}
Approximating $f(\xi, t)$ by Eq. (\ref{eq:kangaroo_gaussian_approx}) and substituting it into Eq. (\ref{eq:kangaroo_correlation_fxn_in_f}) shows that
\begin{eqnarray}
C(t) &\approx& \frac{\bar{\xi}_t^2 \exp(-t \lambda(\bar{\xi}_t))}{\avg{\xi^2}} \int d \xi~  \exp\bigg(-\bigg[2/\bar{\xi}_t^2 +  t \lambda''(\bar{\xi}_t)\bigg] \times (\xi - \bar{\xi}_t)^2\bigg) ~p(\xi) \nonumber  \\
&\approx&  \frac{\bar{\xi}_t^2 \exp(- t \lambda(\bar{\xi}_t))}{\avg{\xi^2}} \frac{\sqrt{\pi}}{\bigg[2/\bar{\xi}_t^2+  t \lambda''(\bar{\xi}_t) \bigg]^{1/2}} \times p(\bar{\xi}_t) \nonumber \\
&\propto& \boxed{ \frac{\bar{\xi}_t^3 \exp(- t \lambda(\bar{\xi}_t))}{\bigg[\bar{\xi}_t^2  \lambda''(\bar{\xi}_t)  t + 2 \bigg]^{1/2}} }~.
\end{eqnarray}