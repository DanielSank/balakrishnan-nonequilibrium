\levelstay{Problem 2.4.1}

\textbf{(a):} Find closed-form expressions for $\tavg{u^{2 l}}$ and $\tavg{u^{2 l+1}}$.

Starting with the even moments, one has that
\begin{eqnarray}
\tavg{u^{2 l} } &=& 4 \pi \int^{\infty}_{0} u^{2 l} ~f^{\text{eq}}(u) u^2 du \nonumber \\
&=& \frac{4}{\sqrt{\pi}} \bigg(\frac{m}{2 k T}\bigg)^{\frac{3}{2}} \cdot \bigg(\frac{2 k T}{m}\bigg)^{\frac{2l+3}{2}} \underbrace{\int^{\infty}_{0} z^{2(l+1)} \exp(-z^2) dz}_{=\Gamma(l+1+1/2)/2} \nonumber \\
&=& \frac{4}{\sqrt{\pi}} \cdot \bigg(\frac{2 k T}{m}\bigg)^l \frac{\sqrt{\pi} ~(2l+1)!}{2^{2l+1} ~ l!} \times \frac{1}{2} \nonumber \\
&=& \boxed{\frac{(2l+1)!}{2^{l} ~ l!} \times \bigg(\frac{k T}{m}\bigg)^l}~.
\end{eqnarray}
Similarly for the odd moments, one finds that
\begin{eqnarray}
\tavg{u^{2 l+1} } &=& 4 \pi \int^{\infty}_{0} u^{2 l+1} ~f^{\text{eq}}(u) u^2 du \nonumber \\
&=& \frac{4}{\sqrt{\pi}} \bigg(\frac{m}{2 k T}\bigg)^{\frac{3}{2}} \cdot \bigg(\frac{2 k T}{m}\bigg)^{\frac{2l+4}{2}} \underbrace{\int^{\infty}_{0} z^{2l+3} \exp(-z^2) dz}_{=\Gamma(l+2)/2} \nonumber \\
&=& \frac{4}{\sqrt{\pi}} \cdot \bigg(\frac{2 k T}{m}\bigg)^{\frac{2l+1}{2}} \frac{(l+1)!}{2} \nonumber \\
&=&  \boxed{\frac{2^{l+\frac{3}{2}} (l+1)!}{\sqrt{\pi}} \times \bigg(\frac{k T}{m}\bigg)^{l+\frac{1}{2}}}~. 
\end{eqnarray}
Here have used the following properties of the $\Gamma$-function
\begin{equation}
\Gamma(r+1/2) = \frac{\sqrt{\pi} (2r-1)!}{2^{2r-1} (r-1)!}~\text{for}~r\in\{1, 2, 3, \dots\},
\end{equation}
\begin{equation}
\Gamma(r+1) = r!~\text{for}~r\in\{0, 1, 2, \dots\},
\end{equation}
and the substitution $z = u \times \sqrt{m/2 k T}$.
Note that these expressions hold for negative exponents $u^{-|m|}$ provided that $|m|<3$, otherwise the integral diverges as $u\rightarrow0$.

\textbf{(b):} Show that
\begin{equation}
\tavg{u^{-1}} > 1/\tavg{u} \nonumber
\end{equation}
and 
\begin{equation}
\tavg{u^{-2}} > 1/\tavg{u^2} \nonumber
\end{equation}
using the Cauchy-Schwartz inequality for $L^2$ functions over the interval $[0, \infty)$.

Starting with the first inequality, define the following two functions:
\begin{equation}
f(u) = \sqrt{4\pi \bigg( \frac{m}{2\pi k T}\bigg)^{3/2} \frac{1}{u}~u^2 \exp\bigg(-\frac{m u^2}{2 k T}\bigg) } \nonumber
\end{equation}
and
\begin{equation}
g(u) = \sqrt{4\pi \bigg( \frac{m}{2\pi k T}\bigg)^{3/2} u ~u^2 \exp\bigg(-\frac{m u^2}{2 k T}\bigg) }. \nonumber
\end{equation}
Notice that
\begin{equation}
\int_{0}^{\infty} f(u)^2 du = \tavg{u^{-1}}, \nonumber
\end{equation}
\begin{equation}
\int_{0}^{\infty} g(u)^2 du  = \tavg{u}, \nonumber
\end{equation}
and
\begin{equation}
\int_{0}^{\infty} f(u)g (u)du  = \int_{-\infty}^{\infty} p^{\text{eq}}(v) dv = 1. \nonumber
\end{equation}
By the Cauchy-Schwarz inequality
\begin{equation}
\bigg|\int_{0}^{\infty} f(u) g(u) du \bigg|^2 \leq \bigg(\int_{0}^{\infty} f^2(u) du \bigg) \bigg(\int_{0}^{\infty} g^2(u) du \bigg),
\end{equation}
which implies that
\begin{equation}
1 < \tavg{u^{-1}} \tavg{u}.
\end{equation}
Rearranging this inequality proves that $\tavg{u^{-1}} > 1/\tavg{u}$ \footnote{We have used a strict inequality here since $f(u)$ and $g(u)$ are linearly independent functions.}

Similarly, we can define 
\begin{equation}
f'(u) = \sqrt{4\pi \bigg( \frac{m}{2\pi k T}\bigg)^{3/2} \frac{1}{u^2}~u^2 \exp\bigg(-\frac{m u^2}{2 k T}\bigg) }
\end{equation}
and
\begin{equation}
g'(u) = \sqrt{4\pi \bigg( \frac{m}{2\pi k T}\bigg)^{3/2} u^2 ~u^2 \exp\bigg(-\frac{m u^2}{2 k T}\bigg) }.
\end{equation}
Notice that
\begin{equation}
\int_{0}^{\infty} f'(u)^2 du = \tavg{u^{-2}},
\end{equation}
\begin{equation}
\int_{0}^{\infty} g'(u)^2 du  = \tavg{u^2},
\end{equation}
and
\begin{equation}
\int_{0}^{\infty} f'(u)g'(u)du  = \int_{-\infty}^{\infty} p^{\text{eq}}(v) dv = 1,
\end{equation}
By the Cauchy-Schwarz inequality, we have that
\begin{equation}
1 < \tavg{u^{-2}} \tavg{u^2}.
\end{equation}
Again, rearranging gives the desired result.