\section{Ch. 4 (AO)}
\subsection{Problem 4.5.2}

\textbf{(a):} From the definition $M_{kj} = \epsilon_{kjl} n_l$, one has that
\begin{equation}
\textbf{M}= \begin{bmatrix}
0 & n_3 & -n_2\\
-n_3 & 0 & n_1 \\
n_2 & -n_1 & 0 
\end{bmatrix}.
\end{equation}
The eigenvalues of this matrix are given by the expression
\begin{equation}
\det(\textbf{M}-\lambda \textbf{1})= 0,
\end{equation}
which implies that
\begin{eqnarray} \begin{vmatrix}
-\lambda & n_3 & -n_2\\
-n_3 & -\lambda & n_1 \\
n_2 & -n_1 & -\lambda 
\end{vmatrix} &=& -\lambda \times \begin{vmatrix}
-\lambda & n_1 \\
-n_1 & -\lambda \end{vmatrix} - n_3 \begin{vmatrix}
-n_3 & n_1 \\
n_2 & -\lambda 
\end{vmatrix} - n_2 \begin{vmatrix}
-n_3 & -\lambda \\
n_2 & -n_1 
\end{vmatrix} \\
&=& -\lambda^3 -\lambda\underbrace{(n_1^2 + n_2 ^ 2 + n_3^2)}_{=1} \\&=&  -\lambda^3 - \lambda = -\lambda(\lambda^2+1) =  0.
\end{eqnarray}
This shows that $\lambda = 0, \pm i$ are the eigenvalues of this matrix. By the Cayley-Hamilton theorem one has that $\textbf{M}^3=-\textbf{M}$ as stated in the book.

We compute $\exp(\pm i \textbf{M} \omega_c t)$ as follows\footnote{Notice that $\textbf{M}^3 = -\textbf{M}$, $\textbf{M}^4 = -\textbf{M}^2$, $\textbf{M}^5 = +\textbf{M}$, $\textbf{M}^6 = +\textbf{M}^2$, and so on.}
\begin{eqnarray}
\exp(\pm i \textbf{M} \omega_c t) &=& \textbf{1} + (\pm \textbf{M} \omega_c t) + \frac{(\pm \textbf{M} \omega_c t)^2}{2!} + \underbrace{\frac{(\pm \textbf{M} \omega_c t)^3}{3!}}_{-\textbf{M} (\pm \omega_c t)^3/3!} +  \underbrace{\frac{(\pm \textbf{M} \omega_c t)^4}{4!}}_{-\textbf{M}^2 (\pm \omega_c t)^4/4!} + \cdots \nonumber \\
&=& \textbf{1} + \textbf{M} \times \bigg((\pm \omega_c t) - \frac{(\pm \omega_c t)^3}{3!} + \frac{(\pm \omega_c t)^5}{5!} + \cdots \bigg) \nonumber \\
&+& \textbf{M}^2\times  \bigg( \frac{(\pm \omega_c t)^2}{2!} -  \frac{(\pm \omega_c t)^4}{4!} +  \frac{(\pm \omega_c t)^6}{6!} + \ldots \bigg) \nonumber \\
&=& \boxed{1 \pm \textbf{M} \sin(\omega_c t) + \textbf{M}^2 (1-\cos(\omega_c t))}.
\end{eqnarray}

\textbf{(b):} The $kj^{\text{th}}$ element of $\textbf{M}^2$ is given by
\begin{eqnarray}
(\textbf{M}^2)_{kj} = M_{kr} M_{rj} &=& \epsilon_{krl} n_l \times \epsilon_{rjs} n_s \nonumber \\
&=& -\epsilon_{rkl} n_l \epsilon_{rjs}  n_s \nonumber \\
&=& -(\delta_{kj} \delta_{ls} - \delta_{ks} \delta_{lj}) n_l n_s \nonumber \\
&=& (\delta_{ks} \delta_{lj}-\delta_{kj} \delta_{ls}) n_l n_s.
\end{eqnarray}
which has implied sums over $l$ and $s$. This term is nonzero only when $(k=s,~l=j)$ or when $(k=j,~l=s)$, meaning that 
\begin{equation}
\boxed{M_{kr} M_{rj} = n_k n_j - \delta_{k,j} (n_1^2 + n_2^2 + n_3^2) = n_k n_j - \delta_{j, k}}~.
\end{equation}

\subsection{Problem 4.5.3}

Eq. (4.34) from the textbook states that
\begin{equation}
C_{ij}(t)= \frac{\tavg{v_i(0) v_j(t)}}{(k_B T/m)} = e^{-\gamma |t|} [ n_i n_j + (\delta_{ij} - n_i n_j) \cos(\omega_c t) - \epsilon_{ijk} n_k \sin(\omega_c t)].
\end{equation}
Computing $\tavg{\vec{v}(0) \vec{v}(t)}$ is equivalent to $\sum_{i}(k_B T/m) \times C_{ii}(t)$, which shows that
\begin{eqnarray}
\tavg{\vec{v}(0) \vec{v}(t)} &=& \frac{k_B T}{m} e^{-\gamma |t|} \bigg\{ (n_1^2 + n_2^2 + n_3^2) + (3 -(n_1^2 + n_2^2 + n_3^2)) \cos(\omega_c t) \bigg\} \\
&=& \boxed{\frac{k_B T}{m} e^{-\gamma |t|} ( 1 + 2 \cos(\omega_c t))}~.
\end{eqnarray}

In a similar manner, the cross product is given by
\begin{eqnarray}
\tavg{\vec{v}(0) \times \vec{v}(t)} &=& \frac{k_B T}{m} \sum_{i} \epsilon_{ijk} C_{jk}(t) \hat{n}_i \nonumber \\
&=& \frac{k_B T}{m} \sum_{i} \sum_{j, k} \epsilon_{ijk} \bigg\{ e^{-\gamma |t|} [ n_j n_k + (\delta_{jk} - n_j n_k) \cos(\omega_c t) - \epsilon_{jks} n_s \sin(\omega_c t)]\bigg\} \hat{n}_i \nonumber \\
&=& \frac{k_B T}{m} \sum_{i} \sum_{j, k} \bigg\{ e^{-\gamma |t|} [ \epsilon_{ijk} n_j n_k + (\underbrace{\epsilon_{ijk} \delta_{jk}}_{=0} -\epsilon_{ijk} n_j n_k) \cos(\omega_c t) - \epsilon_{ijk} \epsilon_{jks} n_s \sin(\omega_c t)]\bigg\}\hat{n}_i \nonumber.
\end{eqnarray}
The term $\sum_{i} \sum_{j, k} = \epsilon_{ijk} n_j n_k = 0$, therefore\footnote{The term $\sum_{i} \sum_{j, k} \epsilon_{jki} \epsilon_{jks}$ equals +1 for two combinations of $(j, k)$ for any fixed value of $i=s$. For example, if $i=s=1$, the combinations $(j=2, k=3)$ and $(j=3, k=2)$ both contribute at +1 in these sums, hence the factor of of 2. }
\begin{eqnarray}
\tavg{\vec{v}(0) \times \vec{v}(t)} &=& \frac{k_B T}{m} \sum_{i} \sum_{j, k} \bigg\{ e^{-\gamma |t|} [-\epsilon_{ijk} \epsilon_{jks} n_s \sin(\omega_c t)]\bigg\}\hat{n}_i \nonumber \\
&=& \frac{k_B T}{m} \sum_{i} \sum_{j, k} \bigg\{ e^{-\gamma |t|} [-\epsilon_{ijk} \epsilon_{jks} n_s \sin(\omega_c t)]\bigg\}\hat{n}_i \\
&=& \frac{k_B T}{m} \sum_{i} \sum_{j, k} \bigg\{ e^{-\gamma |t|} [-\epsilon_{jki} \epsilon_{jks} n_s \sin(\omega_c t)]\bigg\}\hat{n}_i \nonumber \\
&=& \frac{k_B T}{m} \sum_{i}\bigg\{ e^{-\gamma |t|} [-2 \delta_{is} n_s \sin(\omega_c t)]\bigg\}\hat{n}_i \nonumber \\
&=& -\frac{2k_B T}{m} e^{-\gamma |t|} \sin(\omega_c t) \sum_{i} n_i \hat{n}_i \nonumber \\
&=& \boxed{-\frac{2k_B T}{m} e^{-\gamma |t|}\sin(\omega_c t) \vec{n}}~.
\end{eqnarray}