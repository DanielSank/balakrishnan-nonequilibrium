\levelstay{Problem 4.5.1}

\leveldown{Problem}

\begin{enumerate}[a.]
\item Use Eq.~(4.38) to show that
\begin{equation*}
  \angavg{v(t_1) v(t_2) v(t_3)}_\text{eq} = 0 \, .
\end{equation*}
Eq.~(4.38) is
\begin{equation*}
  \baravg{\eta(t_1) \eta(t_2) \eta(t_3)} = 0 \, .
\end{equation*}
\item Use Eq.~(4.40) to show that
\begin{align*}
  \angavg{v(t_1) v(t_2) v(t_3) v(t_4)}_\text{eq}
  = \left( \frac{k_b T}{m} \right) \{
    &\exp \left(-\gamma \abs{t_1 - t_2} - \gamma \abs{t_3 - t_4} \right) \\
  + &\exp \left(-\gamma \abs{t_1 - t_3} - \gamma \abs{t_2 - t_4} \right) \\
  + &\exp \left(-\gamma \abs{t_1 - t_4} - \gamma \abs{t_2 - t_3} \right)
  \}
  \, .
\end{align*}
Eq.~(4.40) is
\begin{align*}
  \baravg{\eta(t_1) \eta(t_2) \eta(t_3) \eta(t_4)} =
  &  \baravg{\eta(t_1) \eta(t_2)} \times \baravg{\eta(t_3) \eta(t_4)} \\
  +& \baravg{\eta(t_1) \eta(t_3)} \times \baravg{\eta(t_2) \eta(t_4)} \\
  +& \baravg{\eta(t_1) \eta(t_4)} \times \baravg{\eta(t_2) \eta(t_3)}
  \, .
\end{align*}
\end{enumerate}

\levelstay{Solution}

First of all, we know that the formal solution for $v(t)$ is
\begin{equation*}
  v(t) = v_0 e^{-\gamma t} + \frac{1}{m} \int_0^t dt' \, \eta(t') \, e^{-\gamma(t - t')}
\end{equation*}
where $v_0$ is the initial velocity and $eta$ is the noise process.
The thermal average $\angavg{}_\text{eq}$ averages over the noise and over the initial velocity, which is given by the Boltzman distribution.
Following the strategy in the book, let's first compute the conditional average, i.e. the average where $v_0$ is considered to be a given sure variable, and then we'll compute the thermal averages from that by allowing $\{t_i\}$ to go to infinity while keeping their differences constant.

Note that the integral of $\eta$ has an even probability distribution so its average is zero, even with the exponential weight factor inside the integral.
This fact causes a bunch of crossterms in the following calculation to go to zero.
Let's create a convenient notation
\begin{equation*}
  v(t_i) = v_0 e^{-\gamma t_i} + \frac{1}{m} \int_0^{t_i} ds_i \eta(s_i) e^{-\gamma(t_i - s_i)}
\end{equation*}
i.e. for each time $t_i$ we use the dummy integration variable $s_i$.

There are four kinds of terms in $\baravg{v(t_1) v(t_2) v(t_3)}$:
\begin{enumerate}[1.]
  \item The term with no integrals:
    \begin{equation*}
      v_0^3 \exp \left( -\gamma \sum_a t_a \right)
    \end{equation*}
    This term goes to zero in the thermal limit as $t_i \rightarrow \infty$.
  \item The three terms with one integral:
    \begin{equation*}
      \frac{v_0^2}{m} \exp \left(-\gamma \sum_a t_a \right) \int_0^{t_i} ds_i \, \eta(s_i) e^{\gamma s_i}
    \end{equation*}
    These terms all go to zero because $\baravg{\eta(s_i)} = 0$.
  \item The three terms with two integrals:
    \begin{equation*}
	    \frac{v_0}{m^2} e^{-\gamma t_i} \left[ e^{-\gamma (t_j + t_k)} \int_0^{t_j} ds_i \int_0^{t_k} ds_j \, ds_k \,  \eta(s_j) \eta(s_k) e^{\gamma (s_j + s_k)} \right]
    \end{equation*}
    where the three terms correspond to the three permutations of $\{1, 2, 3 \}$.
    The thermal average of the factor in square brackets is computed on page 32 of the book where the result is
    \begin{equation*}
      \frac{k_b T}{m} e^{-\gamma \abs{t_j - t_k}} \, .
    \end{equation*}
    That factor doesn't vanish, but the prefactor $(v_0 / m^2) e^{-\gamma t_i}$ does vanish because $t_i \rightarrow \infty$ causes the exponential to go to zero.
  \item The term with three integrals:
    \begin{equation*}
      \frac{e^{-\gamma \sum_a t_a}}{m^3} \prod_{a=1}^3 \int_0^{t_a} ds_a \, e^{\gamma s_a} \eta(s_a)
    \end{equation*}
		The conditional (and therefore thermal) average of this expression is zero because $\baravg{\prod_{a=1}^3 \eta(s_a)} = 0$.
\end{enumerate}
We've shown that all terms in $\angavg{v(t_1) v(t_2) v(t_3)}$ are zero, which proves part a.

Part b is pretty much the same thing. Here again there is a variety of terms but we've learned from part a that only the term with four integrals needs consideration.
\begin{equation*}
    \frac{1}{m^4} e^{-\gamma \sum_a t_a} \prod_{a=1}^4 \int_0^{t_a} ds_a \, e^{\gamma s_a} \eta(s_a)
\end{equation*}
Now we use the fact that
\begin{equation*}
  \baravg{\prod_{a=1}^4 \eta(s_a)} = \Gamma^2 \sum_\text{comb} \delta(s_i - s_j)\delta(s_k - s_l)
\end{equation*}
where $\sum_\text{comb}$ means that we sum over all unique pairings of the numbers $\{1, 2, 3, 4\}$.
Using this fact, our quadruple integral term becomes
\begin{equation*}
  \frac{1}{m^4} \sum_\text{comb}
  \left( e^{-\gamma (t_i + t_j)}\int_0^{t_i} ds_i \int_0^{t_j} ds_j \, e^{\gamma ( s_i + s_j )} \eta(s_i) \eta(s_j) \right)
  \left( e^{-\gamma (t_k + t_l)}\int_0^{t_k} ds_k \int_0^{t_l} ds_l \, e^{\gamma ( s_k + s_l )} \eta(s_k) \eta(s_l) \right)
  \, .
\end{equation*}
The factors in parantheses are the same integral we discussed above (from page 32 of the book) so we can just plug in their values
\begin{equation*}
  \frac{1}{m^4} \sum_\text{comb}
  \left( \frac{k_b T}{m} \right)^2 e^{-\gamma \abs{t_i - t_j}} e^{-\gamma \abs{t_k - t_l}}
\end{equation*}
which is what we set out to show.
