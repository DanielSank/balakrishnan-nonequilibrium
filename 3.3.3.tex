\levelstay{Problem 3.3.2}

\leveldown{Problem}

This is a slightly harder problem.
Instead of the very specific decaying exponential form in Eq.~(3.18), suppose you are given only that
\begin{itemize}
  \item $\baravg{\eta(t_1)\eta(t_2)}$ is a function of $\abs{t_1-t_2}$, the magnitude of the difference between the two time arguments; and further,
  \item $\baravg{\eta(t_1)\eta(t_2)}$ tends to zero as $\abs{t_1 - t_2} \rightarrow \infty$.
\end{itemize}
Show that the inescapable conclusion based on these properties is that $\baravg{v^2(t)}$ increases linearly with $t$ at long times, in the absence of $\gamma$.

\levelstay{Solution}

Denote $K(\abs{t_1 - t_2}) \equiv \baravg{\eta(t_1)\eta(t_2)}$.
In the absence of friction, the Lengevin equation reads
\begin{equation*}
  \baravg{v(t)^2} = \frac{1}{m^2} \int_0^t dt_1 \int_0^t dt_2 K(\abs{t_1 - t_2}) \, .
\end{equation*}
Figure~\ref{fig:3.3.3} illustrates our strategy for doing the integral.
\begin{align*}
  m^2 \baravg{v(t)^2}
  &= \int_0^t dt_1 \int_0^t dt_2 K(\abs{t_1 - t_2}) \equiv \mathcal{I}(t) \\
  (\text{Fig.~\ref{fig:3.3.3}\,a}) \quad &= \int_\text{(A + B)} K(\abs{t_1 - t_2}) \\
  \text{(by symmetry)} \quad &= 2 \int_\text{A} K(\abs{t_1 - t_2}) \\
  &= 2 \int_0^t dt_1 \int_0^{t_1} dt_2 \, K(\abs{t_1 - t_2}) \\
  (\text{Fig~\ref{fig:3.3.3}\, b: Let }t' \equiv t_1 - t_2) \quad &= 2 \int_0^t dt_1 \int_0^{t_1} dt' \, K(t')
  \, .
\end{align*}
This last integral is over a right triangle where we scan the horizontal axis first, and then the vertical one.
It's more convenient to make $t'$ the first integration variable, so we instead integrate over vertical axis (see Fig.~\ref{fig:3.3.3}\,c):
\begin{equation*}
  \mathcal{I}(t) = \int_0^t dt' \, K(t') \int_{t'}^t dt_1 = \int_0^t dt' \, K(t') \, (t - t')
  \, .
\end{equation*}

Now while the exercise says to assume that $\lim_{t \rightarrow \infty} K(t) = 0$, that's actually neither a sufficient nor necessary assumption for what we're supposed to be proving here.
In fact, the author even gives in Ch 15 the case of $K(x) \sim 1/x$ as a counterexample where the integrals don't even converge.
Furthermore, the author makes several somewhat vague statements about superdiffusive and subdiffusive behavior, without really commenting on how those behaviors relate to the long time behavior of $K$.
In this solution set, we do not tolerate such Textbook Terrorism.
Instead, we do things right.

A much more reasonable assumption to make about $K$ is that it's integral approaches a finite constant $K^*$.
To make that statement specific, we assume that given any $\epsilon > 0$, there is a time $T$ such that
\begin{equation*}
  \abs{\int_0^t K(t') \, dt' - K^*} < \epsilon
\end{equation*}
for all $t > T$.
In other words, the integral of $K$ can be as close to $K^*$ as you want as long as you integrate out far enough.
Now let's take a look at that integral $\mathcal{I}(t)$.
Naturally, we split it into two pieces
\begin{equation*}
  \mathcal{I}(t) = \int_0^T dt' \, K(t') (t - t') + \int_T^t dt' \, K(t')(t - t')
  \, .
\end{equation*}
We're interested in it's scaling behavior at large times so we study $d\mathcal{I}/dt$:
\begin{align*}
	\frac{d\mathcal{I}}{dt}
	&= \frac{d}{dt}
	\left(
		t \int_0^t dt' \, K(t') - \int_0^t dt' \, t' K(t')
	\right) \\
	&= t K(t) + \int_0^t dt' \, K(t') - t K(t) \\
	&= \int_0^t dt' \, K(t') \\
	&\approx K^*
\end{align*}
where the last approximation holds to arbitary accuracy for sufficiently large $t$.
We've shown that for sufficiently large $t$, $d\mathcal{I}/dt$ is approximately constant, which means that $\mathcal{I}(t)$ is increasing linearly in $t$, as we set out to show.


\quickfig{\columnwidth}{3.3.3}{a) The original integral is over the region indicated by the black square, i.e. A + B. The integrand depends only on $\abs{t_1 - t_2}$ so the integrals over regions $A$ and $B$ are equal. b) After the change of variables $t' = t_1 - t_2$, we are integrating over a triangle (similar to region A) where the vertical axis is the argument of the $K$ function. We integrate over $t$ first and then $t'$, i.e. $t$ is the outer integration variable. c) We switch so that $t'$ is the outer integration variable.}{fig:3.3.3}
