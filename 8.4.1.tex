\levelstaynon{8.4.1 - Solution by separaiton of variables: Special Edition}

\leveldownnon{Problem}

Solve the diffusion equaiton on the interval $[-a a]$ with reflecting boundary conditions.
For the fun of it, we solve for the case where the initial condition is
\begin{equation*}
  p(x, t) = \delta(x - x_0) \delta(t - t_0)
\end{equation*}
i.e. we allow the initial delta distribution to be off-center, breaking the symmetry of the problem that Balki gave us.
We're interested in this case because the solution has a form that may be surprising to those familiar with Fourier series.

\levelstaynon{Solution}

We are solving the equation
\begin{equation*}
  \left( \frac{\partial}{\partial t} - D \left( \frac{\partial}{\partial x} \right)^2 \right)
  p = \delta_{x_0} \delta_{t_0}
\end{equation*}
subject to the condition
\begin{equation*}
  \left( \frac{\partial}{\partial x} p \right)(x\in \{-a, a\}, t) = 0
  \, .
\end{equation*}

Consider a solution of the form
\begin{equation*}
  p(x, t) = \sum_n X_n(x) T_n(t)
\end{equation*}
where the range of the integers $n$ is not yet determined.
As the boundary conditions pertain only to $X_n$ we focus on that part first.
The eigenvalue equation for $X$ is
\begin{equation*}
  X''_n + \lambda_n X = 0
\end{equation*}
which is solved by
\begin{equation*}
  X_n(t) = c_n \cos(k_n x) + s_n \sin(k_n x)
\end{equation*}
The boundary conditions are then
\begin{eqnarray*}
  - c_n k_n \sin(k_n a) + s_n k_n \cos(k_n a) &= 0 \\
  \text{and} \qquad
  - c_n k_n \sin(-k_n a) + s_n k_n \cos(-k_n a) &= 0
  \, .
\end{eqnarray*}
The sum and difference of these equations yield two conditions on $c_n$, $s_n$, and $k_n$:
\begin{itemize}
  \item For each $n$ where $c_n \neq 0$, $k_n = (\pi / 2 a) (2 n)$.
  \item For each $n$ where $s_n \neq 0$, $k_n = (\pi / 2 a) (1 + 2n)$.
\end{itemize}
From these conditions, it follows logically that for each $n$ only one of $c_n$ or $s_n$ can be nonzero.
But that is suspicious because for $n=1$ we get two functions, $\cos(\pi x / a)$ and $\sin(3 \pi x / 2 a)$ which are orthogonal and should therefore both be permissible in the solution.
This is happening because our presumed form of $X_n$ is ``wrong'' in the sense that it's not an eigenvector of the $X$ equation, i.e. because the sine and cosine parts have different numbers of cycles over the interval and therefore have different eigenvalues.
Fortunately, it's clear how to fix the problem.
Notice that over the interval $[-a a]$ the cosines have an integer number of cycles while the sines have half-integer number of cycles (remember that the length of the interval is $2 a$.).
Therefore, let's enumerate the solutions by the number of half-cycles by rewriting all of them as cosines ($c_n$ here is not the same $c_n$ as above -- we're redefining it):
\begin{equation*}
  X_n(x) = \cos \left( \left[ \frac{\pi x}{2 a} + \frac{\pi}{2} \right] n \right)
  \, .
\end{equation*}
This enumeration over shifted cosines runs over the same set of functions as our previous enumeration over unshifted sines and cosines.
You can easily check that for even $n$ we get the cosines (with $-1$ factors in front that can be absorbed into our new definition of $c_n$) and for odd $n$ we get the sines.
Now, each $X_n$ is an eigenfunction of $(\partial / \partial x)^2$ with $\lambda_n = - (n \pi / 2 a)^2$.

We could have just stated the shifted cosines as the original choice of $X_n$ and saved some text, so why didn't we?
I chose to write the solution in the way we find here because \emph{this is how I figured it out in real life}.
In fact, I spent weeks trying to solve this problem by starting with a Fourier series
\begin{equation*}
  \sum_{n=0}^\infty c_n \cos(2 \pi n x / 2 a) + s_n \sin(2 \pi n x / 2 a)
\end{equation*}
which doesn't work because (I think) it makes the wrong assumptions about the boundary conditions.
It was only after retreating from my presupposed form of the spatial functions that I finally figured out how to solve this problem, and only after struggling to understand why the original form of $X_n$ written above doesn't work.

Anyway, now that we have a good form of $X_n$ we can continue.
We solve for the time dependence via Fourier transformation.
Express $T_n$ in the Fourier domain:
\begin{equation*}
  p(x, t)
  = \int \frac{d\omega}{2\pi} \sum_{n=0}^\infty
    e^{i \omega t} \tilde{T}_n(\omega) X_n(x)
  \, .
\end{equation*}
Now apply $(\partial / \partial t) - D (\partial / \partial x)^2$ to each side to get
\begin{equation*}
  \delta(x - x_0) \delta(t - t_0)
    = \dot{T}_0(t) + \int \frac{d \omega}{2\pi}
    \sum_{n=1}^\infty (i \omega + D (\pi n / 2a)^2)
    \tilde{T}_n(\omega) e^{i \omega t}
    X_n(x)
  \, .
\end{equation*}
where we put just the $n=0$ term back into the time domain because it simplifies the next step.
We're going to solve for $T_0$ first, and then $T_n$ for $n \geq 1$.
To find $T_0$, integrate both sides over $x$ to get (noting that the integral over $x$ of $X_n$ is zero)
\begin{equation*}
  \delta(t - t_0) = (2 a) \dot{T}_0(t)
\end{equation*}
with solution
\begin{equation*}
  T_0(t) = \frac{1}{2 a} \Theta(t - t_0)
  \, .
\end{equation*}
To find the other $T_n$, multiply both sides of the equation by $X_m$ and integrate over $x$ to get (noting that the integral over $x$ of $X_n X_m$ is $a \delta_{m, n}$)
\begin{equation*}
  X_m(x_0) \delta(t - t_0)
  = a \int \frac{d\omega}{2\pi} (i\omega + D(\pi m / 2a)^2) \tilde{T}_m(\omega) e^{i \omega t} \\
\end{equation*}
and then multiply both sides by $\exp(-i \omega' t)$ and integrate over $t$ to get
\begin{align*}
  X_m(x_0) e^{-i \omega' t_0}
  &= a (i \omega' + D(\pi m / 2 a)^2) \tilde{T}_m(\omega') \\
  \Rightarrow \tilde{T}_m(\omega)
  &= \frac{1}{a} \frac{X_m(x_0) \exp(-i \omega t_0)}{i \omega + D (\pi m / 2 a)^2} \\
  &= \frac{-i}{a} \frac{X_m(x_0) \exp(-i \omega t_0)}{\omega - i D (\pi m / 2 a)^2} \\
  T_m(t) = \int \frac{d\omega}{2\pi} e^{i \omega t} \tilde{T}_m(\omega)
  &= \frac{-i}{a} X_m(x_0) \int \frac{d\omega}{2\pi} \frac{\exp(i \omega (t - t_0))}{\omega - i D (\pi m / 2 a)^2} \\
  (\text{contour integration}) \quad \Rightarrow
  T_m(t) &= \frac{1}{a} X_m(x_0) \exp \left( -D (\pi m / 2 a)^2 (t - t_0) \right)
  \, .
\end{align*}
So finally the solution is
\begin{equation*}
  p(x, t) = \frac{1}{2a} \Theta(t - t_0)
  + \frac{1}{a} \sum_{n=1}^\infty X_n(x) X_n(x_0) \exp \left( -D \left( \frac{\pi n}{2 a} \right)^2 (t - t_0) \right)
\end{equation*}

A Python script providing an animation of this solution is provided in \textbf{exercise\_841.py} included in the source repository of this document.
