\levelstay{Problem 7.5.1}

\leveldown{Problem}

Solve the diffusion equation
\begin{equation*}
  \left( \frac{\partial}{\partial t} - D \frac{\partial^2}{\partial x^2} \right) p(x, t) = 0 \, .
\end{equation*}

\levelstay{Solution}

We solve this problem using Fourier transforms only, no Laplace, and we do it in a way that
\begin{itemize}
    \item Illuminates the meaning of the Laplace transform, and
    \item sets us up nicely for the next exercise by solving for the Green's function of the diffusion equation.
\end{itemize}
In the next exercise, we'll introduce a notation that makes thinking about this type of problem much, much easier.

Balki sure likes the Laplace transform, but I find it somewhat mysterious, especially because I don't understand its formal inverse.
Furthermore, I find it confusing that Balki's solution to the Diffusion equation, Eq.~(7.11) in the text, isn't just zero for all $x$ and $t$, because he didn't include any source terms or boundary conditions in the original statement of the problem (Eq.~(7.1)).
I'm going to address these confusions head on.

Start with the diffusion equation \emph{including a source term}
\begin{equation*}
  \left( \frac{\partial}{\partial t} - D \frac{\partial^2}{\partial x^2} \right) p(x, t) = J(x, t) \, .
\end{equation*}
The function $J(x,t)$ indicates addition of probability density at various times and places.
Of course, $p(x, t)$ can only be interpreted as a probability if the integral over all $x$ of $J(x, t)$ is zero for every $t$, as otherwise we'd be creating or destroying probability.
In the case that $J$ is not restricted in this way, we can just think of $p$ as an un-normalized population distribution instead of a probability.
On the other hand, allowing for the destruction of probability is completely reasonable if we want to consider e.g. a case where the diffusing particle can be lost.

Anyway, for this problem let's consider the case where $J(x, t) = \delta(x - x_0) \delta(t - t_0)$, because that's what Balki is doing implicitly when he uses the Laplace transform solution.
This choice of $J$ means that we're creating 1 unit of probability at position $x_0$ at time $t_0$.

The probability $p(x, t)$ can be expressed as a Fourier transform
\begin{equation*}
  p(x, t) = \int \frac{dk}{2\pi} \int \frac{d \omega}{2\pi} \tilde{p}(k, \omega)
  e^{i k x} e^{i \omega t}
\end{equation*}
Plugging this Fourier transform into the diffusion equation, we get
\begin{equation*}
  (i \omega + D k^2)\tilde{p}(k, \omega) = \tilde{J}(k, \omega)
\end{equation*}
and noting that
\begin{equation*}
  \tilde{J}(k, \omega) = \exp \left(-i (k x_0 + \omega t_0) \right)
\end{equation*}
we have
\begin{equation*}
  \tilde{p}(k, \omega) = \frac{e^{-i(kx_0 + \omega t_0)}}{i \omega + D k^2} \, .
\end{equation*}
Then
\begin{align*}
  p(x, t)
  &= \int \frac{dk}{2\pi}\int \frac{d\omega}{2\pi}
    \frac{e^{i k (x - x_0)} e^{i \omega (t - t_0)}}{i \omega + D k^2} \\
  &= -i \int \frac{dk}{2\pi} \, e^{i k (x - x_0)}
    \int \frac{d \omega}{2\pi} \frac{e^{i \omega (t - t_0)}}{\omega - i D k^2}
  \, .
\end{align*}
The integral over $\omega$ can be done using contour integration.
There's a pole at $\omega = i D k^2$ which we pick up if we close the contour in the upper half plane, which is appropriate when $t > t_0$.
On the other hand, when $t < t_0$ we close the contour in the lower half plane, there's no pole, and so $p(x, t < t_0) = 0$, which makes sense because for $t < t_0$ the source term hasn't injected any probability.
Doing the contour integral for the $t > t_0$ case we get
\begin{align*}
  p(x, t)
  &= \int \frac{dk}{2 \pi} e^{i k (x - x_0) - D k^2 (t - t_0)} \\
  &= \frac{1}{\sqrt{4 \pi D (t - t_0)}} \exp \left( - \frac{(x - x_0)^2}{4 D (t - t_0)} \right)
  \, .
\end{align*}
If we choose $x_0 = 0$ and $t_0 = 0$ then we get what Balki wants us to show.
Note, of course, that by solving this problem with a more general source, we've derived exactly Balki's Green's function from Eq.~(7.26).
