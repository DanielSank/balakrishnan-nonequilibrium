\section{Ch. 2 (AO)}
\subsection{Problem 2.4.1}

\textbf{(a):} Find closed-form expressions for $\tavg{u^{2 l}}$ and $\tavg{u^{2 l+1}}$.

Starting with the even moments, one has that
\begin{eqnarray}
\tavg{u^{2 l} } &=& 4 \pi \int^{\infty}_{0} u^{2 l} ~f^{\text{eq}}(u) u^2 du \nonumber \\
&=& \frac{4}{\sqrt{\pi}} \bigg(\frac{m}{2 k T}\bigg)^{\frac{3}{2}} \cdot \bigg(\frac{2 k T}{m}\bigg)^{\frac{2l+3}{2}} \underbrace{\int^{\infty}_{0} z^{2(l+1)} \exp(-z^2) dz}_{=\Gamma(l+1+1/2)/2} \nonumber \\
&=& \frac{4}{\sqrt{\pi}} \cdot \bigg(\frac{2 k T}{m}\bigg)^l \frac{\sqrt{\pi} ~(2l+1)!}{2^{2l+1} ~ l!} \times \frac{1}{2} \nonumber \\
&=& \boxed{\frac{(2l+1)!}{2^{l} ~ l!} \times \bigg(\frac{k T}{m}\bigg)^l}~.
\end{eqnarray}
Similarly for the odd moments, one finds that
\begin{eqnarray}
\tavg{u^{2 l+1} } &=& 4 \pi \int^{\infty}_{0} u^{2 l+1} ~f^{\text{eq}}(u) u^2 du \nonumber \\
&=& \frac{4}{\sqrt{\pi}} \bigg(\frac{m}{2 k T}\bigg)^{\frac{3}{2}} \cdot \bigg(\frac{2 k T}{m}\bigg)^{\frac{2l+4}{2}} \underbrace{\int^{\infty}_{0} z^{2l+3} \exp(-z^2) dz}_{=\Gamma(l+2)/2} \nonumber \\
&=& \frac{4}{\sqrt{\pi}} \cdot \bigg(\frac{2 k T}{m}\bigg)^{\frac{2l+1}{2}} \frac{(l+1)!}{2} \nonumber \\
&=&  \boxed{\frac{2^{l+\frac{3}{2}} (l+1)!}{\sqrt{\pi}} \times \bigg(\frac{k T}{m}\bigg)^{l+\frac{1}{2}}}~. 
\end{eqnarray}
Here have used the following properties of the $\Gamma$-function
\begin{equation}
\Gamma(r+1/2) = \frac{\sqrt{\pi} (2r-1)!}{2^{2r-1} (r-1)!}~\text{for}~r\in\{1, 2, 3, \dots\},
\end{equation}
\begin{equation}
\Gamma(r+1) = r!~\text{for}~r\in\{0, 1, 2, \dots\},
\end{equation}
and the substitution $z = u \times \sqrt{m/2 k T}$.
Note that these expressions hold for negative exponents $u^{-|m|}$ provided that $|m|<3$, otherwise the integral diverges as $u\rightarrow0$.

\textbf{(b):} Show that
\begin{equation}
\tavg{u^{-1}} > 1/\tavg{u} \nonumber
\end{equation}
and 
\begin{equation}
\tavg{u^{-2}} > 1/\tavg{u^2} \nonumber
\end{equation}
using the Cauchy-Schwartz inequality for $L^2$ functions over the interval $[0, \infty)$.

Starting with the first inequality, define the following two functions:
\begin{equation}
f(u) = \sqrt{4\pi \bigg( \frac{m}{2\pi k T}\bigg)^{3/2} \frac{1}{u}~u^2 \exp\bigg(-\frac{m u^2}{2 k T}\bigg) } \nonumber
\end{equation}
and
\begin{equation}
g(u) = \sqrt{4\pi \bigg( \frac{m}{2\pi k T}\bigg)^{3/2} u ~u^2 \exp\bigg(-\frac{m u^2}{2 k T}\bigg) }. \nonumber
\end{equation}
Notice that
\begin{equation}
\int_{0}^{\infty} f(u)^2 du = \tavg{u^{-1}}, \nonumber
\end{equation}
\begin{equation}
\int_{0}^{\infty} g(u)^2 du  = \tavg{u}, \nonumber
\end{equation}
and
\begin{equation}
\int_{0}^{\infty} f(u)g (u)du  = \int_{-\infty}^{\infty} p^{\text{eq}}(v) dv = 1. \nonumber
\end{equation}
By the Cauchy-Schwarz inequality
\begin{equation}
\bigg|\int_{0}^{\infty} f(u) g(u) du \bigg|^2 \leq \bigg(\int_{0}^{\infty} f^2(u) du \bigg) \bigg(\int_{0}^{\infty} g^2(u) du \bigg),
\end{equation}
which implies that
\begin{equation}
1 < \tavg{u^{-1}} \tavg{u}.
\end{equation}
Rearranging this inequality proves that $\tavg{u^{-1}} > 1/\tavg{u}$ \footnote{We have used a strict inequality here since $f(u)$ and $g(u)$ are linearly independent functions.}

Similarly, we can define 
\begin{equation}
f'(u) = \sqrt{4\pi \bigg( \frac{m}{2\pi k T}\bigg)^{3/2} \frac{1}{u^2}~u^2 \exp\bigg(-\frac{m u^2}{2 k T}\bigg) }
\end{equation}
and
\begin{equation}
g'(u) = \sqrt{4\pi \bigg( \frac{m}{2\pi k T}\bigg)^{3/2} u^2 ~u^2 \exp\bigg(-\frac{m u^2}{2 k T}\bigg) }.
\end{equation}
Notice that
\begin{equation}
\int_{0}^{\infty} f'(u)^2 du = \tavg{u^{-2}},
\end{equation}
\begin{equation}
\int_{0}^{\infty} g'(u)^2 du  = \tavg{u^2},
\end{equation}
and
\begin{equation}
\int_{0}^{\infty} f'(u)g'(u)du  = \int_{-\infty}^{\infty} p^{\text{eq}}(v) dv = 1,
\end{equation}
By the Cauchy-Schwarz inequality, we have that
\begin{equation}
1 < \tavg{u^{-2}} \tavg{u^2}.
\end{equation}
Again, rearranging gives the desired result.

\subsection{Problem 2.4.2}

Find an expression for the PDF in terms of the particles kinetic energy $\varepsilon = m u ^2 / 2$. 

The integral of the velocity PDF in spherical coordinates is given by
\begin{equation}
1 = \frac{4}{\sqrt{\pi}} \bigg(\frac{m}{2 k T}\bigg)^{3/2} \int_{0}^{\infty} u^2 \exp\bigg(-\frac{m u^2}{2 k T}\bigg).
\end{equation}
Making the change of variables 
\begin{eqnarray*}
\varepsilon &=& \frac{1}{2} m u ^2 \\
u &=& \sqrt{\frac{2}{m}} \varepsilon \\
du &=& \sqrt{\frac{2}{m}}~\frac{1}{2}~\frac{d \varepsilon}{\sqrt{\varepsilon}},
\end{eqnarray*}
gives
\begin{eqnarray*}
1 &=& \frac{4}{\sqrt{\pi}} \bigg(\frac{m}{2 k T}\bigg)^{3/2} \int_{0}^{\infty} \frac{2}{m}~ \varepsilon~\exp\bigg(\frac{-\varepsilon}{k T}\bigg) ~ \sqrt{\frac{2}{m}}~\frac{1}{2}~\frac{d \varepsilon}{\sqrt{\varepsilon}} \\
&=& \frac{2}{\sqrt{\pi}} \bigg(\frac{2}{m}\bigg)^{3/2} \bigg(\frac{m}{2 k T}\bigg)^{3/2} \int_{0}^{\infty} \sqrt{\varepsilon} ~ \exp\bigg(\frac{-\varepsilon}{k T}\bigg)~d \varepsilon\\
&=& \frac{2}{\sqrt{\pi}} \bigg(\frac{1}{k T}\bigg)^{3/2} \int_{0}^{\infty} \sqrt{\varepsilon}~ \exp\bigg(\frac{-\varepsilon}{k T}\bigg)~d \varepsilon.
\end{eqnarray*}
Identifying the integrand of this expression as the desired PDF yields
\begin{equation}
\boxed{\phi^{\text{eq}}(\varepsilon) = \frac{2}{\sqrt{\pi}} \bigg(\frac{1}{k T}\bigg)^{3/2} \sqrt{\varepsilon} \exp\bigg(\frac{-\varepsilon}{k T}\bigg)}~.
\end{equation}

\subsection{Problem 2.4.3}

Consider two distinct particles of mass $m$ immersed in a fluid that is in thermal equilibrium. We assume that both particles move completely independently of each other, and that they do not interact in any manner. Let $v_1$ and $v_2$ be their respective velocities. \textbf{Goal:} Find the normalized PDF of the relative velocity $v_\text{rel} \equiv v_1-v_2$.

Denoting this PDF by $F^\text{eq}(v_\text{rel})$, the formal expression for this function is
\begin{eqnarray}
F^\text{eq}(v_\text{rel}) &=& \int_{-\infty}^{\infty} dv_1 \int_{-\infty}^{\infty} dv_2 ~ p^\text{eq}(v_1) ~ p^\text{eq}(v_2) ~ \delta(v_\text{rel} - (v_1-v_2)) \nonumber \\
&=& \bigg( \frac{m}{2\pi k T}\bigg) \int_{-\infty}^{\infty} dv_1 ~ \exp\bigg(\frac{-m v_1^2}{2 k T}\bigg)~\exp\bigg(\frac{-m (v_1-v_\text{rel})^2}{2 k T}\bigg) \nonumber \\
&=& \bigg( \frac{m}{2\pi k T}\bigg) \exp\bigg(-\frac{m v_\text{rel}^2}{2 k T}\bigg) \int_{-\infty}^{\infty} dv_1 ~\exp\bigg(\frac{-m (v_1^2-v_1 v_\text{rel})}{k T}\bigg). \nonumber
\end{eqnarray}
In the second line of this equation, we have used the fact that the $\delta(v_\text{rel} - (v_1-v_2))$ is nonzero when $v_2=v_1-v_\text{rel}$, thus collapsing the integral over $dv_2$.
Using the formula
\begin{equation}
\int_{-\infty}^{\infty} \exp(-a x^2 + bx)~dx = \sqrt{\frac{\pi}{a}} \exp\bigg(\frac{b^2}{4a}\bigg), \nonumber
\end{equation}
where $a=m/k T$ and $b = a v_\text{rel}$, one has that
\begin{eqnarray}
F^\text{eq}(v_\text{rel}) &=& \bigg( \frac{m}{2\pi k T}\bigg) \sqrt{\frac{\pi k T}{m}} \exp\bigg(-\frac{m v_\text{rel}^2}{2 k T}\bigg) \times \exp\bigg(+\frac{m v_\text{rel}^2}{4 k T}\bigg) \nonumber \\
&=& \boxed{\bigg( \frac{m}{4\pi k T}\bigg)^{1/2} ~ \exp\bigg(-\frac{m v_\text{rel}^2}{4 k T}\bigg)}~.
\end{eqnarray}
This means that the PDF of the relative velocity is also a Gaussian, but with twice the variance of the single particle case. Note that we cannot use a spherically symmetric representation of the single-particle PDFs in this problem. This function is properly normalized since an intergal over $d v_\text{rel}$ could have been carried out first, thus eliminating the $\delta$-function for arbitrary values of $v_1$ \& $v_2$.


\subsection{Problem 2.4.4}

Consider $n$ tagged particles of mass $m$ immersed in a fluid that is in thermal equilibrium. Let us assume that the concentration of these particles is vanishingly small, and furthermore, that they do not interact with one another. Under these assumptions, each particle moves independently of the others. Let $V = (v_1 + \ldots + v_n)/n$ be the velocity of their center of mass. \textbf{goal:} find the PDF of $V$ which we denote by $F^\text{eq}(V)$.

The PDF is given by
\begin{equation}
F^\text{eq}(V) = \int_{-\infty}^{\infty} dv_1 \cdots \int_{-\infty}^{\infty} dv_n ~ p^\text{eq}(v_1) \cdots p^\text{eq}(v_n) ~ \delta(V - \frac{1}{n}\sum_{i=1}^{n} v_i).
\end{equation}
Using the Fourier representation of the $\delta$-function
\begin{equation}
\delta(x) = \frac{1}{2\pi} \int_{-\infty}^{\infty} dp ~ \exp(i p x),
\end{equation}
we focus on the integral of 
\begin{equation}
\int_{-\infty}^{\infty} dv_1 \exp\bigg(-\frac{m v_1^2}{2 k T} -i p v_1/n\bigg) =  \bigg( \frac{2\pi k T}{m}\bigg)^{1/2} \times \exp\bigg(-\frac{p^2 k T }{2 n^2 m}\bigg).
\end{equation}
Notice that each of the $p^\text{eq}(v_j)$ terms comes with a normalization factor $\sqrt{m/2 \pi k T}$ that cancels with the first term in the expression above. This leaves us with one factor of $\exp(-p^2 k T/2 n^2 m)$ per particle. Repeating this process for all remaining particles and restoring the missing factor of $1/2\pi$ from the Fourier representation of the $\delta$-function yields the following integral:
\begin{eqnarray}
F^\text{eq}(V) &=& \frac{1}{2\pi}\int_{-\infty}^{\infty} dp ~ \exp\bigg(-\frac{p^2 k T }{2 n^2 m} \times n + i p V\bigg)\\
&=&  \boxed{\sqrt{\frac{m n}{2\pi k T}} ~ \exp\bigg(-\frac{V^2 m n }{2 k T} \bigg)}~.
\end{eqnarray}
This is means that the PDF of a linear combination of independent Gaussian random variables is again a Gaussian. Notice that the width of this distribution narrows with increasing $n$.